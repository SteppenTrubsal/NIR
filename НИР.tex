\documentclass[a4paper,12pt]{article}

\usepackage{vkriate}

% \setFRMfontfamily{cmr}
% \setFRMdfontfamily{ptm}
\setFRMdfontsize{10}

% задает длину поля для подписи на титульной странице
\newFRMfield{xtitlesign}{32mm}

% поле для факультета или кафедры
\newFRMfield{fcath}{65mm}

%имя файла с библиографией в формате BibTex
\addbibresource{rbiblio.bib}

\begin{document}

% счетчики страниц, рисунков, таблиц
\regtotcounter{page}
\regtotcounter{figure}
\regtotcounter{table}

\renewcommand{\refname}{\centerline{СПИСОК ИСПОЛЬЗОВАННЫХ ИСТОЧНИКОВ}} 
\renewcommand{\contentsname}{\centerline{СОДЕРЖАНИЕ}} 
%\renewcommand{\refname}{Список источников}  % По умолчанию "Список литературы" (article)
%\renewcommand{\bibname}{Литература}  % По умолчанию "Литература" (book и report)

\thispagestyle{empty}

\begin{center}\small
\textbf{МИНИСТЕРСТВО НАУКИ И ВЫСШЕГО ОБРАЗОВАНИЯ РОССИЙСКОЙ ФЕДЕРАЦИИ}\\
ФЕДЕРАЛЬНОЕ ГОСУДАРСТВЕННОЕ АВТОНОМНОЕ ОБРАЗОВАТЕЛЬНОЕ УЧРЕЖДЕНИЕ
ВЫСШЕГО  ОБРАЗОВАНИЯ\\
«Национальный исследовательский ядерный университет «МИФИ»\\
\textbf{Обнинский институт атомной энергетики} – \\
филиал федерального государственного автономного образовательного учреждения высшего\\
образования «Национальный исследовательский ядерный университет «МИФИ»\\
(ИАТЭ НИЯУ МИФИ)
\end{center}

\medskip

\begin{center}
\begin{tabular}{rl}
Отделение &
\useFRMfield{fcath}[\large Интеллектуальные кибернетические системы] \\
Направление подготовки &
\useFRMfield{fcath}[\large Информатика и вычислительная техника] \\
\end{tabular}
\end{center}

\vfill

\begin{center}\large
Научно-исследовательская работа

\medskip

\textbf{\Large
Анализ кодогенераторов для CANopen
}
\end{center}

\vspace{1cm}

\begin{center}
\begin{tabular*}{\textwidth}{lcr}
Студент группы ИВТ-Б22 &
\useFRMfield{xtitlesign} &
Карасев Н. А. \\
& & \\
Руководитель & & \\
инженер-программист &
\useFRMfield{xtitlesign} &
Жильцов Д. И.
\end{tabular*}
\end{center}

\vfill
\large

\begin{center}
Обнинск, 2025 г
\end{center}

\onehalfspacing
\pagebreak

\section*{\centering РЕФЕРАТ}

Работа \total{page} стр., \total{table} табл.,
\total{figure} рис., \totalmycitecounts\ ист.
\pagebreak


\tableofcontents
\pagebreak

% Допускается определения, обозначения и сокращения приводить в одном структурном элементе «ОПРЕДЕЛЕНИЯ, ОБОЗНАЧЕНИЯ И СОКРАЩЕНИЯ».

\section*{\centering ТЕРМИНЫ И ОПРЕДЕЛЕНИЯ}


\pagebreak

\section*{\centering ПЕРЕЧЕНЬ СОКРАЩЕНИЙ И ОБОЗНАЧЕНИЙ}

\pagebreak

\tocsection{ВВЕДЕНИЕ}

\tocsection{ЗАКЛЮЧЕНИЕ}

\pagebreak

\addcontentsline{toc}{section}{СПИСОК ИСПОЛЬЗОВАННЫХ ИСТОЧНИКОВ}
\printbibliography

\pagebreak

\renewcommand{\appendixpagename}{\centering Приложения}

%\begin{appendices}

%\renewcommand{\thesection}{\Asbuk{section}}

%\makeatletter
%\renewcommand{\theProgram}{\thesection.\@arabic\c@Program}
%\makeatother

%\section{\centering}
%\setcounter{Program}{0}

%\begin{flushleft}
%\needspace{3\baselineskip}
%\captionof{Program}{Часть кода реализации класса HashMapValue}\label{app1}
%\begin{MyCodes}
%public class HashMapValue {
%    protected String filename;
%    protected HashMap<String, String> hashValue = new HashMap<>();
%    protected HashMap<String, Boolean> hashKeysFlag = new HashMap<>();
%}
%\end{MyCodes}
%\end{flushleft}

%\pagebreak

%\begin{flushleft}
%\captionof{Program}{Пример кода}\label{app2}
%\begin{MyCodes}
%код второго приложения
%\end{MyCodes}
%\end{flushleft}

%\end{appendices}

\end{document}
